
\chapter*{Abstract}

This thesis studies the application of representations based on compact data structures on a practical problem: analyzing massive collections of historic passenger trajectories over transportation networks. These networks can be either from public transportation (such as buses or trains) or a network formed by streets (as in the case taxis or bicycles).

In the first part, we introduce compact data representations specifically designed for the storage of indexed collections of trajectory data, for both public transportation and urban street network contexts. We also prove that these representations can be exploited to answer a diverse set of queries using efficient algorithms.

In the second part, we discuss the design and implementation of an end-to-end platform to support several use cases related to the analysis of trajectories over public transportation networks, integrating our compact representations and providing a graphical query interface with the use of spatial visualization technologies.

\chapter*{Resumen}

Esta tesis estudia la aplicaci\'on de las representaciones basadas en estructuras de datos compactas en un problema pr\'actico: el an\'alisis de grandes colecciones de movimientos hist\'oricos de pasajeros sobre redes de transporte. Estas redes pueden ser de transporte p\'ublico (como autobuses o trenes) o tambi\'en redes formadas por carreteras (como en el caso de los taxis o bicicletas).

En la primera parte, introducimos representaciones de datos compactas espec\'ificamente dise\~nadas para el almacenamiento colecciones indexadas de datos de trayectorias, tanto para el contexto del transporte p\'ublico como el de redes de calles. Tambi\'en demostramos que estas representaciones pueden ser explotadas para responder un diverso conjunto de consultas utilizando algoritmos eficientes.

En la segunda parte, tratamos el dise\~no y la implementaci\'on de una plataforma end-to-end para soportar varios casos de uso relacionados con el an\'alisis de trayectorias sobre redes de transporte p\'ublico, integrando nuestras representaciones compactas y proporcionando una interfaz gr\'afica de consultas mediante el uso de tecnolog\'ias de visualizaci\'on espacial.

\chapter*{Resumo}

Esta tese estuda a aplicaci\'on de representaci\'ons baseadas en estruturas compactas de datos nun problema pr\'actico: a an\'alise de grandes colecci\'ons de movementos hist\'oricos de pasaxeiros nas redes de transporte. Estas redes poden ser de transporte p\'ublico (como autobuses ou trens) ou tam\'en redes formadas por estradas (como no caso de taxis ou bicicletas).

Na primeira parte, introducimos representaci\'ons de datos compactos deseñadas especificamente para colecci\'ons indexadas de almacenamento de datos de traxectoria, tanto para o contexto do transporte p\'ublico como para as redes de r\'uas. Tam\'en mostramos que estas representaci\'ons poden ser explotadas para responder a un conxunto diverso de consultas usando algoritmos eficientes.

Na segunda parte, trataremos o deseño e implementaci\'on dunha plataforma end-to-end para soportar varios casos de uso relacionados coa an\'alise de traxectorias nas redes de transporte p\'ublico, integrando as nosas representaci\'ons compactas e proporcionando unha interface gr\'afica de consultas a trav\'es do uso de tecnolox\'ias de visualizaci\'on espacial.

 

