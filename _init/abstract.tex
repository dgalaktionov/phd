
\chapter*{Abstract}

The proliferation of GPS devices in smartphones, vehicles and sport wearables on the one hand, and geolocation mechanisms (such as smart cards in public transportation) on the other hand, have led to an unprecedented ability to gather and store trajectories that originate from people's movements during their daily schedules. However, no standard data models exist to represent these trajectories and, in addition, neither traditional databases nor new \textit{NoSQL} databases are adequate for the representation and exploitation of the complex \mbox{spatio-temporal} data that make up such trajectories. This general outlook is even more complex once we consider that whenever we are storing information related to the context of public transportation passengers, customers inside a mall, or simply vehicles moving in a city we must deal with a true Big Data scenario in which guaranteeing an efficient response can be very challenging.

Consequently, in this thesis we address the design of compact data structures for the representation of the followed trajectories, both in the context of vehicles and/or people moving in urban or periurban spaces, as in the context of itineraries of commuters in public transportation. Apart from designing these compact data structures that allow us to represent the Big Data scenario usually seen in this application domain, we have also designed the algorithms that allow the efficient exploitation of the underlying information.

We have implemented algorithms that not only to solve the classic \mbox{spatio-temporal} queries, such as obtaining the position of a moving object at a time instant, reconstructing the trajectory of an object, or even \mbox{spatio-temporal} window queries (which objects are inside a spatial range either within a time window or at a time instant), but also solve more specialized queries for the analysis of the trajectories that travelers make. For instance, we have designed algorithms to query the number of travelers that start (or finish) their trip in a certain place within a given time interval, or the number of travelers that switch from one line from the public transportation network to another one using a particular stop, or even the number of travelers that had started their trip in a certain place (which can be either a stop or a whole neighborhood) and finished it in another place.

Both the designed structures and the querying algorithms, which are available at \url{https://github.com/dgalaktionov/compact-trip-representation}, have been experimentally evaluated. With these structures we were able to represent, in a compact space of 100 MiB, a collection of approximately a million and a half of taxi trajectories, or alternatively ten million trajectories consisting of itineraries over public transportation networks (the latter being more compressible). In both cases, we can solve most of the considered exploitation queries in the order of microseconds, with algorithms that scale logarithmically with respect to the increase in the number of stored trajectories.

Finally, considering that this work is considered an industrial thesis, and that this requires showing that the research performed is of clearly applied nature, we have developed a web application with Geograhic Information Systems technology, which integrates with our compressed structures and algorithms instead of relying on common spatial databases. This application, which provides a simple and intuitive user interface that represents the map of a transportation network, enabled an end user to run the aforementioned querying algorithms over a large collection of historic trajectories. Likewise, this interface presents the query results in a graphical and intuitive way.

\chapter*{Resumen}

La proliferaci\'on de por un lado de dispositivos GPS en smartphones, veh\'iculos o pulseras de deporte,  y por otro, de otros mecanismos de geolocalizaci\'on (como las tarjetas de pago de trasporte p\'ublico), han dado lugar a una capacidad in\'edita de obtener y almacenar las trayectorias que generan las personas al moverse durante sus quehaceres diarios. Sin embargo, no existen modelos de datos est\'andar para representar dichas trayectorias, adem\'as de que ni las bases de datos tradicionales, ni las nuevas bases de datos \textit{NoSQL} se adec\'uan bien a la representaci\'on y explotaci\'on de esos datos complejos de naturaleza \mbox{espacio-temporal} que son las trayectorias.  Para hacer m\'as complejo a\'un el panorama, se constata adem\'as que cuando se quieren almacenar trayectorias de viajeros de transporte p\'ublico, o de clientes en centros comerciales, o simplemente de personas o veh\'iculos movi\'endose por una ciudad hay que enfrentarse a un verdadero escenario Big Data en el que la eficiencia en la respuesta a las consultas se hace muy dif\'icil.

Por todo ello, en esta tesis se aborda el dise\~no de estructuras de datos compactas para la representaci\'on de las trayectorias seguidas, por un lado, por veh\'iculos y/o personas que se mueven por las calles de un entorno urbano o periurbano acotado, y por otro los itinerarios de viajeros de transporte p\'ublico. Adem\'as de dise\~nar esas estructuras de datos compactas, que permiten representar ese escenario Big Data habitual en estos dominios de aplicaci\'on, se han dise\~nado los algoritmos que permiten la explotaci\'on eficiente de dichos datos.

Hemos implementado algoritmos que, adem\'as de resolver las consultas \mbox{espacio-temporales} cl\'asicas, tanto las de posici\'on de un objeto en un tiempo, o trayectoria de un objeto durante un intervalo temporal, como las consultas de rango \mbox{espacio-temporal} (qu\'e objetos est\'an en una ventana del espacio en un instante o intervalo temporal) resuelven tambi\'en consultas m\'as especializadas para el an\'alisis de trayectorias de viajeros. Por ejemplo, hemos dise\~nado algoritmos para  consultar el n\'umero de viajeros que inician (o terminan)  su viaje en un lugar dado dentro de un cierto intervalo temporal, o el n\'umero de viajeros que conmutan de una l\'inea a otra de la red de transporte p\'ublico en una parada concreta, o incluso el n\'umero de viajeros que inicia su viaje en cierto lugar (parada o  barrio) y lo termina en otra parada o barrio determinados. 

Tanto las estructuras de datos dise\~nadas como todos los algoritmos de consulta,  que  est\'an disponibles en \url{https://github.com/dgalaktionov/compact-trip-representation},  han sido evaluados experimentalmente. Con estas estructuras es posible representar en un espacio de 100 MiB una colecci\'on de aproximadamente un mill\'on y medio de trayectorias de taxis, o alternativamente diez millones de trayectorias consistentes de itinerarios sobre redes de transporte p\'ublico, al ser \'estas \'ultimas m\'as compactas. En ambos casos, podemos resolver la mayor parte de las consultas de explotaci\'on planteadas en el orden de microsegundos, con algoritmos que escalan de forma logar\'itmica con respecto al incremento en el n\'umero de trayectorias almacenadas.

Por \'ultimo, considerando que este trabajo est\'a considerado como una tesis industrial, lo cual requiere demostrar que el trabajo investigador realizado es de naturaleza aplicada, hemos desarrollado una aplicaci\'on web con tecnolog\'ia de Sistemas de Informaci\'on Geogr\'afica que, en vez de trabajar sobre una base de datos espacial convencional, utiliza la estructura comprimida y los algoritmos para su explotaci\'on dise\~nados en la tesis. Esa aplicaci\'on facilita, mediante una sencilla e intuitiva interfaz de usuario que representa el mapa de la red de transporte, el lanzamiento de los algoritmos dise\~nados sobre un amplio conjunto de trayectorias de viajeros. Del mismo modo esa interfaz presenta los resultados de las consultas de modo gr\'afico e intuitivo.

\chapter*{Resumo}

A proliferaci\'on de por un lado dos dispositivos GPS en smartphones, veh\'iculos ou brazaletes deportivos e por outra banda dos mecanismos de xeolocalizaci\'on (como as tarxetas de pago do transporte p\'ublico), te\~nen dado lugar a unha capacidade sen precedentes para obter e almacenar as traxectorias que a xente xera ao moverse durante as s\'uas tarefas diarias. Sen embargo, non hai modelos de datos est\'andar para representar ditas traxectorias, e ademais de que nin as bases de datos tradicionais nin as novas bases de datos \textit{NoSQL} se adec\'uan ben \'a representaci\'on e explotaci\'on dos datos tan complexos e de natureza espazo-temporal que son as traxectorias. Para complicar a\'inda m\'ais o panorama, tam\'en se comproba que cando se queren almacenar traxectorias de viaxeiros de transporte p\'ublico, ou de clientes en centros comerciais, ou simplemente de persoas ou veh\'iculos que se desprazan por unha cidade, se ten que afrontar un verdadeiro escenario de Big Data no que a eficiencia na resposta \'as consultas se fai moi dif\'icil.

Por iso, esta tese trata do dese\~no de estruturas compactas de datos para a representaci\'on dos cami\~nos seguidos, por un lado, por veh\'iculos e/ou persoas que se desprazan polas r\'uas dun contorno urbano ou periurbano delimitado, e por outro lado os itinerarios de viaxeiros en transporte p\'ublico. Ademais de dese\~nar estas estruturas compactas de datos, que permiten representar dito escenario Big Data habitual nestes dominios de aplicaci\'on, dese\~n\'aronse algoritmos que permiten a explotaci\'on eficiente dos devanditos datos.

Estes algoritmos, ademais de resolver as cl\'asicas consultas espazo-temporais, tanto a posici\'on dun obxecto nun instante dado, como a traxectoria dun obxecto durante un intervalo de tempo, as\'i como as consultas de rango espazo-temporal (que obxectos est\'an nun rango do espazo nun intre ou nun intervalo temporal) tam\'en permiten resolver consultas m\'ais especializadas para a an\'alise de traxectorias de viaxeiros. Por exemplo, dese\~namos algoritmos para comprobar o n\'umero de viaxeiros que inician (ou rematan) a s\'ua viaxe nun determinado lugar nun certo intervalo de tempo, ou o n\'umero de viaxeiros que cambian dunha li\~na a outra da rede de transporte p\'ublico nunha parada concreta, ou incluso o n\'umero de viaxeiros que comezan a s\'ua viaxe nun determinado lugar (parada ou barrio) e rematan noutra parada ou barrio espec\'ifico.

Tanto as estruturas de datos dese\~nadas como todos os algoritmos de consulta, dispo\~nibles en \url{https://github.com/dgalaktionov/compact-trip-representation}, foron avaliados experimentalmente. Con estas estruturas \'e posible representar nun espazo de 100 MiB unha colecci\'on de aproximadamente un mill\'on e medio de traxectos de taxi ou, alternativamente, dez mill\'ons de traxectos consistentes en itinerarios en redes de transporte p\'ublico, por ser estes \'ultimos m\'ais compactos. Nos dous casos, podemos resolver a maior\'ia das consultas de explotaci\'on plantexadas na orde de microsegundos, con algoritmos que escalan logar\'itmicamente con respecto ao aumento do n\'umero de traxectorias almacenadas.

Finalmente, dado o car\'acter de tese industrial deste traballo, foi necesario que a investigaci\'on realizada tivese un car\'acter claramente aplicado, polo que se implementou unha aplicaci\'on web con tecnolox\'ia de Sistemas de Informaci\'on Xeogr\'afica que, no canto de traballar nunha base de datos espacial convencional, usa a estrutura comprimida e algoritmos de explotaci\'on dese\~nados na tese. Esta aplicaci\'on facilita, mediante unha interface de usuario sinxela e intuitiva que representa o mapa da rede de transporte, o lanzamento dos algoritmos dese\~nados nun amplo conxunto de rutas de pasaxeiros. Do mesmo xeito, dita interface presenta os resultados das consultas dun xeito gr\'afico e intuitivo.
 
