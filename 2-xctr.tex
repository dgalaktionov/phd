	This is about a representation that was designed for buses.
	
	Here we speak about TTCTR and XCTR, where we have a network description and index journeys instead of explicit times.
	
	\section{Description}
	A network representation, then CSA for stops. In TTCTR we encode lines in these stops, while in XCTR we make the twist from before (I think) and then separate lines into a WM. Whatever the case, we then have an aligned WM of journeys. Oh and T-Matrices as a DW-like structure.
	
	\section{Algorithms}
	A bit complex for both TTCTR and XCTR. Maybe copy the complexity table from the paper that explains why TTCTR sucks for some queries and we needed to develop XCTR.
	
	\section{Experiments}
	Here we compare one to the other. Maybe include query times for postgresql, maybe later.