\chapter{Resumen del trabajo realizado}
\label{ch:appendix-spanishsummary}

    En este cap\'itulo se presenta un resumen del trabajo realizado durante la tesis. En la secci\'on \ref{sec:appendix-spanishsummary:introduccion} se puede encontrar una introducci\'on, donde describimos brevemente el problema a resolver. En la secci\'on \ref{sec:appendix-spanishsummary:conclusiones} resumimos todas las contribuciones de esta tesis. Finalmente, en la \'ultima secci\'on \ref{sec:appendix-spanishsummary:trabajo-futuro} mencionamos los futuros desarrollos planteados como continuaci\'on del trabajo expuesto.

\section{Introducci\'on}
\label{sec:appendix-spanishsummary:introduccion}

\subsection{Motivaci\'on}

    En el contexto de las redes de transporte p\'ublico, los \'ultimos a\~nos han visto numerosos avances en las tecnolog\'ias inal\'ambricas, redes de sensores (especialmente aquellas relacionadas con RFID) y computaci\'on ubicua, lo que ha llevado a una adopci\'on generalizada de la tecnolog\'ia de seguimiento de pasajeros por parte de los servicios de transporte p\'ublico, lo que hace que la recolecci\'on de grandes cantidades de datos sobre los h\'abitos de viaje de estos pasajeros sea m\'as f\'acil que nunca antes.
    Esto, a su vez, ha abierto la puerta a la explotaci\'on de este tipo de informaci\'on para estudiar la demanda (uso) de una red, a diferencia de las t\'ecnicas conocidas para analizar la oferta (rutas, horarios, etc.).
    Para estas aplicaciones, no son los datos sobre trayectorias individuales los que tienen importancia, sino que son las medidas del uso de la red, con las cuales se pueden desarrollar plataformas para el monitoreo del tr\'afico y las tareas de planificaci\'on de carreteras. Ejemplos de medidas \'utiles son los indicadores de accesibilidad y centralidad, referidos a la facilidad para llegar a determinadas ubicaciones o la importancia de ciertas paradas dentro de una red \cite{Morency2007193, El-Geneidy2011, Wang2015335}. Todas estas medidas se basan en alg\'un tipo de consultas de conteo que determinan el n\'umero de distintos viajes que ocurren dentro de una ventana espacial y/o temporal.
    
    Para habilitar estos nuevos tipos de estudios de demanda, es imperativo desarrollar mecanismos que hagan posible persistir y administrar eficientemente estas vastas (y siempre crecientes) colecciones de datos. Cuando tambi\'en tenemos en cuenta que los patrones de consulta eficientes deben ser compatibles para que estos datos sean `` \'utiles '', la soluci\'on claramente constituye un desaf\'io tecnol\'ogico emergente, que se est\'a abordando desde varios dominios diferentes y cientos de soluciones ad-hoc han sido implementados por todas las \textit{Smart Cities} en todo el mundo.
    
    Por consiguiente, una representaci\'on pr\'actica para esta informaci\'on que soporte la indexaci\'on eficiente tendr\'ia numerosas aplicaciones posibles. En \cite{tu2018spatial} podemos ver c\'omo es posible combinar trayectorias GPS con datos de \gls{afc} para recrear trayectorias completas y estudiar la cantidad de pasajeros por \'area. Alternativamente, en \cite{weng2018mining} las trayectorias completas se infieren de los datos \gls{afc}, para luego analizar los patrones de comportamiento y las preferencias de los viajeros con el objetivo de mejorar la eficiencia de la red. Otra aplicaci\'on que se habilita mediante dicho an\'alisis es la publicidad dirigida \cite{zhang2017targeted}, ya que los intereses de un usuario pueden ser perfilados por sus patrones de viaje. Otros trabajos se centran en analizar el uso de paradas o estaciones individuales, como \cite{ceapa2012avoiding}, donde los autores determinan que los tiempos de congesti\'on en la red de metro de Londres son predecibles y ocurren en intervalos de tiempo estrechos. Armado con dicha informaci\'on, un usuario puede elegir un patr\'on de viaje diferente para evitar la multitud y mejorar su experiencia general. Cuando consideramos el transporte p\'ublico a trav\'es de una red de carreteras, podemos encontrar obras centradas en el estudio de los pasajeros de taxis. Un ejemplo notable es \cite{yuan2013t}, que analiza un sistema de recomendaci\'on de taxis bidireccionales, en el que los taxistas se\~nalan los espacios de estacionamiento m\'as rentables mientras los pasajeros son dirigidos a los segmentos de la calle con una alta probabilidad de encontrar un taxi vacante.
    
    Una observaci\'on clave de todos los trabajos mencionados anteriormente es que una simple colecci\'on de trayectorias o puntos marcados en el tiempo sobre un espacio bidimensional de latitud y longitud no ser\'ia lo suficientemente rica como para realizar estos estudios. Por lo tanto, deben trabajar con una representaci\'on que considere cierto grado de informaci\'on \textit{sem\'antica}. Como m\'inimo, esa informaci\'on debe incluir referencias a elementos de la red (paradas, l\'ineas o calles) y, a veces, incluso alg\'un identificador de usuario, pudiendo \'este \'ultimo ser an\'onimo. Por lo tanto, requerimos una representaci\'on que difiera de los \'indices y bases de datos espaciales tradicionales, ya que debe admitir m\'etodos de acceso eficientes basados en elementos de red.


\subsection{Definici\'on del problema}

    Considerando la red subyacente, hemos identificado dos contextos distinguibles para las redes de transporte:
    
    \begin{itemize}
    	\item \textbf{Redes basadas en calles urbanas}: dentro de estas redes, una trayectoria puede comenzar en cualquier momento y ubicaci\'on, y puede seguir cualquier ruta arbitraria de segmentos a lo largo de una red de carreteras definida.
        Las trayectorias de taxis, bicicletas o flotas de veh\'iculos entran en esta categor\'ia. Para estos sistemas, las consultas de inter\'es pueden involucrar puntos de inter\'es alrededor de los cuales podr\'ian terminar estas trayectorias, o segmentos de carreteras que podr\'ian ser parte de una ruta.
        
    	\item \textbf{Redes de transporte p\'ublico}: las trayectorias deben comenzar en puntos predefinidos (por lo general, paradas o estaciones) en los horarios establecidos definidos por los veh\'iculos que realizan una parada en esos puntos. Estos veh\'iculos siguen caminos predefinidos a lo largo de estos puntos, formando rutas, y los usuarios que viajan en el mismo veh\'iculo producir\'ian partes id\'enticas de trayectorias.
    Esta clasificaci\'on se aplica a los sistemas de autobuses y metro, junto con la mayor\'ia de otros sistemas de transporte urbano. Se espera que para una colecci\'on de trayectorias de una red de transporte p\'ublico algunas de las consultas de inter\'es puedan girar en torno a los elementos principales de la red, que son rutas y paradas.
    \end{itemize}
    
    Independientemente del contexto de trabajo, operaremos con un modelo de red, que tiende a ser bastante simple en las redes de calles urbanas, consistiendo simplemente en un grafo dirigido con segmentos de calles como nodos, donde la conexi\'on a otros segmentos de calles indica que La navegaci\'on es posible. Para las redes de transporte p\'ublico, podr\'ia ser pertinente considerar una representaci\'on m\'as rica que un gr\'afico de paradas y l\'ineas, pero considerando tambi\'en las rutas formadas por veh\'iculos de transporte, como autobuses o trenes, que visitan las paradas en horarios establecidos en las que los viajeros pueden subir o bajar en el veh\'iculo. Un modelo muy conocido que incluye estos elementos de red, entre otros menos interesantes para nuestro problema, es el \gls{gtfs} \footnote{\url{https://developers.google.com/transit/gtfs/}}, que ha sido ampliamente adoptado por las plataformas de datos abiertos en numerosas \textit{smart cities}.
    
    Para cualquier tipo de red, una \textit{trayectoria} se definir\'a como una ruta formada por una secuencia de elementos de red (generalmente paradas o segmentos de calles), que fue recorrida por un solo viajero en un viaje, con un origen y un destino final. En esta definici\'on debemos considerar algunas limitaciones pr\'acticas a la naturaleza de una trayectoria, ya que uno podr\'ia argumentar si los viajeros que tardan m\'as de una hora en cambiar una l\'inea realmente las est\'an cambiando, o si simplemente han finalizado sus trayectorias y est\'an comenzando una segunda trayectoria con alg\'un nuevo destino. Estos casos son complicados para decidir inequ\'ivocamente en la pr\'actica y, por lo tanto, nuestro enfoque tender\'a a establecer l\'imites en los tiempos de espera y las distancias de camino entre paradas para una sola trayectoria.
    
    Nuestras definiciones tambi\'en requieren poder hablar por un concepto de tiempo. Cuando trabajamos con un modelo de red de transporte p\'ublico que integra recorridos formados por veh\'iculos de transporte que siguen l\'ineas, no ser\'ia necesario representar la hora exacta en que cada usuario se ha subido a una parada, solo un identificador de recorrido, ya que los tiempos de parada estar\'ian ya disponibles en nuestra red modelada, evitando cierta redundancia en la representaci\'on de trayectorias.
    Alternativamente, para redes de calles urbanas u otros casos donde la informaci\'on de ruta no est\'a disponible, se puede considerar una representaci\'on de intervalos de tiempo para lograr una representaci\'on compacta, donde el tiempo se expresar\'ia en intervalos discretos entre uno y treinta minutos.
    
    Existen t\'ecnicas de recolecci\'on masiva de datos para los dos contextos de red discutidos anteriormente, lo que lleva al problema de manejar eficientemente esta vasta cantidad de informaci\'on. Adem\'as de las soluciones habituales conocidas de \textit{Big Data}, hay una investigaci\'on en curso sobre la aplicaci\'on de estructuras de datos para algunos de estos objetivos. En particular, es posible aplicar muchas de las t\'ecnicas desde el campo de \textit{Compact Data Structures} para crear representaciones autoindexadas que admitan patrones de consulta eficientes adaptados a las necesidades de informaci\'on espec\'ificas, al tiempo que ofrece alg\'un tipo de compresi\'on con respecto a una m\'as tradicional representaci\'on.
    
    Una soluci\'on usable tambi\'en requerir\'ia una interfaz de usuario que permita la explotaci\'on de esta informaci\'on por parte de investigadores, empresas de transporte, administraciones municipales y cualquier otro tipo de usuario final. Esta interfaz debe, como m\'inimo, permitir visualizar los elementos de la red en un mapa, adem\'as de permitir la capacidad de realizar consultas sobre estos elementos de una manera intuitiva y receptiva, respetando los principios de calidad habituales de cualquier software orientado a usuario similar.


%\section{Metodolog\'ia}
%
%
%
%En esta tesis se han realizado una serie de estudios sobre el an\'alisis y el procesamiento de grandes conjuntos de datos en diversas areas. El planeamiento seguido para la realizaci\'on de este trabajo ha sido el siguiente:
%
%\begin{itemize}
%	\item Inicialmente, se realiz\'o un estudio bibliogr\'aficos sobre compresi\'on de datos y estructuras de datos compactos. El objetivo de esta parte es adquirir los mayores conocimientos posibles sobre este campo y sus conceptos b\'asicos, adem\'as conocer el estado de arte actual de las estructuras de datos compactas.
%	\item Una vez adquirido las nociones b\'asicas sobre el campo de la compresi\'on de datos, se realiz\'o un an\'alisis sobre trayectorias de procesos estoc\'asticos de tiempo continuo y como abordar este problema. Al tratarse de secuencias de n\'umero flotantes, nos centramos en el estudio espec\'ifico de la bibliograf\'ia sobre compresi\'on de secuencias de n\'umero flotantes y las m\'etodos utilizadas para dicho prop\'osito. Se pueden distinguir dos estrategias principales. La primera se basa en t\'ecnicas de compresi\'on con p\'erdidas y la segunda intenta predecir un valor utilizando los valores previos de la secuencia y almacena la diferencia entre el valor predicho y el valor real, la cual ser\'a un n\'umero mucho m\'as peque\~no. En esta tesis est\'a enfocada en el dise\~no estrategias sin p\'erdida, por lo tanto, nos centramos en la segunda de ellas.
%	\item En este campo hemos trabajado en el caso particular de las de trayectorais de movimiento Browniando, desarrollando una nueva estructura compacta para su representaci\'on. Esta nueva estructura incorpora mecanismos y datos precalculado para mejorar la velocidad de c\'alculo de la funci\'on de autocovarianza. Esta propuesta se ha validada midiento tanto el tiempo como el espacio consumido durante el c\'alculo de la autocovarianza y compar\'andolos contra la versi\'on del paquete $R$ y nuestra propia versi\'on en $C$.
%	\item Durante el proceso de dise\~no de la estructura anterior, se ha propuesto un nuevo algoritmo 
%\end{itemize}


\section{Contribuciones y conclusiones}
\label{sec:appendix-spanishsummary:conclusiones}

    En los sistemas de transporte de las \textit{smart cities}, las nuevas tecnolog\'ias como el monitoreo del tr\'afico, la recolecci\'on autom\'atica de tarifas (por ejemplo, tarjetas inteligentes) y el conteo autom\'atico de pasajeros han permitido generar una gran cantidad de informaci\'on altamente detallada, datos en tiempo real \'utiles para definir medidas que caracterizan una red de transporte. Estos datos resultan particularmente interesantes porque consisten de viajes reales, combinando as\'i impl\'icitamente la demanda del sistema con la red de calles o el servicio ofrecido por un sistema de transporte p\'ublico.

    Al mismo tiempo, el an\'alisis de estos registros hist\'oricos sobre la demanda de la red puede interesae a las agencias de transporte, as\'i como a otras entidades que puedan estar interesadas en estudiar los movimientos de las multitudes en un contexto urbano. Para hacer posible este tipo de an\'alisis, hemos propuesto representaciones eficaces y flexibles basadas en estructuras de datos compactas que pueden adaptarse f\'acilmente tanto para contextos de transporte urbano (calles y la mayor\'ia de los sistemas de transporte p\'ublico), que pueden manejar diversos casos de uso con considerable eficiencia temporal y de memoria.
    
    Adem\'as, hemos desarrollado una interfaz \gls{gis} que integra las representaciones compactas desarrolladas, lo que permite analizar la informaci\'on almacenada por un usuario final de una manera simple, sin requerir que tengan un conocimiento previo de las tecnolog\'ias subyacentes. Adem\'as, esta interfaz de usuario valida a\'un m\'as la elecci\'on de un enfoque basado en estructuras de datos compactas, demostrando as\'i que es posible desarrollar un producto competitivo que pueda ser adoptado por una organizaci\'on.
    
    Finalmente, en el contexto de la investigaci\'on de estructuras de datos compactas, creemos que este trabajo puede abrir un nuevo campo de aplicaciones pr\'acticas para estas estructuras y algoritmos, como se ha hecho anteriormente para los campos de \textit{recuperaci\'on de informaci\'on} y \textit{bioinform\'atica}.

    La siguiente lista resume las principales contribuciones en nuestro trabajo:

    \begin{enumerate}
        \item En el contexto de viajes a trav\'es de redes de calles urbanas, hemos desarrollado una representaci\'on basada en un \gls{csa} modificado y diferentes alternativas del \gls{wt} llamado \gls{ctr}, que requiere tan solo un 36\% del tama\~no de una l\'inea base sin comprimir, al tiempo que maneja consultas espacio-temporales en el orden de varios \micro segundos, al tiempo que ofrece trade-offs configurables.
        
        \item Hemos propuesto un modelo extensible para la representaci\'on de viajes a trav\'es de una red de transporte p\'ublico, que puede adaptarse para la mayor\'ia de los sistemas de transporte del mundo.
        
        \item Hemos desarrollado dos representaciones alternativas para el contexto de viajes a trav\'es de redes de transporte p\'ublico, \gls{ttctr} y \gls{xctr}, basadas en las mismas estructuras de \gls{ctr}, que pueden resolver la mayor\'ia de nuestras consultas propuestas acerca de los patrones de trayectorias conscientes de la red en el orden de varios \micro segundos, mientras que solo se necesita alrededor del 50\% de espacio en comparaci\'on con una representaci\'on tradicional sin \'indice.
        
        \item Hemos presentado un esquema para comprimir \gls{sat} sin afectar la complejidad temporal de sus operaciones, y lo aplicamos en \gls{tm} como una estructura para acelerar las consultas de carga de red en el contexto del transporte p\'ublico.
        
        \item Hemos desarrollado una interfaz de usuario para analizar la demanda de redes de transporte p\'ublico en funci\'on de nuestras representaciones propuestas, as\'i como las tecnolog\'ias \gls{gis}.
    \end{enumerate}


\section{Trabajo futuro}
\label{sec:appendix-spanishsummary:trabajo-futuro}
    Si bien hay muchos desarrollos futuros posibles para nuestras representaciones propuestas para viajes a trav\'es de redes de transporte p\'ublico, nos preocupa principalmente encontrar una representaci\'on \'unica que pueda manejar eficientemente ambos tipos de consultas propuestas en la secci\'on~\ref{sec:newctr:desc}: las consultas sobre carga de la red y tambi\'en aquellas sobre los patrones de viaje. Encontrar tal representaci\'on resulta considerablemente complejo, ya que la mayor\'ia de las soluciones que permiten agregar eficientemente datos multidimensionales (como paradas y horarios o viajes) no pueden soportar la mayor\'ia de nuestras consultas de patrones de viaje.

    Por otro lado, nuestra interfaz propuesta a\'un est\'a en desarrollo, donde nuestros objetivos m\'as urgentes son soportar una visualizaci\'on intuitiva de l\'ineas que pueden permitir al usuario realizar consultas sobre ellas, as\'i como la adici\'on de medios para consultar el endpoint del histograma, para mostrar el uso de una parada o una l\'inea a lo largo del tiempo con gr\'aficos din\'amicos. Finalmente, esta interfaz se puede mejorar para admitir diferentes visualizaciones interactivas para el uso total de la red en todas las l\'ineas dentro de un rango de tiempo, adem\'as de las consultas de accesibilidad, donde a partir e una parada de la red estamos interesados en obtener el tiempo promedio necesario para llegar a cualquier otra parada.