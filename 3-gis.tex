\part{GIS interface}
\chapter{Previous concepts}
    In order to develop a usable interface, we must rely on a map based representation that works with geographic information. Therefore, in this chapter we shall introduce the basic concepts of \gls{gis} in order to provide a context for this part of our contribution.

    \section{GIS interfaces}
	Because I have a web interface with leaflet after all.

\chapter{GIS interface for public transportation networks}
\label{sec:gis}
	While in the previous part of this work we have presented compact representations that are capable of efficiently handling large collections of trips over both streets and public transportation networks, in order for it to be actually usable by city management and network administrators a front-end interface must be conceived.
	
	In order to facilitate this kind of analysis, we have developed a proof-of-concept map-based interface based on \gls{gis} and web technologies, that allows to exploit some of the capabilities of \gls{xctr} to visualize and query transport demand information over a real transportation network. This constitutes a tool is designed to aid the decision-making process for a public transportation company, and can be easily adapter for different networks and use cases.
	
	In this chapter, we will begin by contextualizing the use case in more detail. After that, we provide an overview of our architecture, both from a functional perspective and a technical one. After that, we will present our API for querying the underlying structure, and finally our prototype user interface.
	
	\section{Motivation}
    Collecting historic locations and movement patterns is currently a hot topic, as more mobile phone corporations, car companies, and other organizations are trying to get advantage on the market through these modern initiatives. While users are tracked and benefit of their services (avoiding traffic congestion, locating restaurants and monuments, etc. on Google Maps; getting the best route on public transport to go somewhere or getting real-time schedule for a bus/train line, etc) those companies are also obtaining valuable data that can be exploited to obtain a profit or to improve the assignment of their resources (e.g., personalized recommendations and advertising, estimating the number of passengers and consequently the number of trains or buses that are required to support all the passengers that will use a line today, etc).
    
    If we focus in public transport, such as buses or trains, we find that most transport companies have been focused mainly on providing helpful information to their passengers regarding the existing offer, hence providing non only information related to their transport network (e.g. maps with the lines, their schedule, etc.) but also, in many cases, real-time information with the actual position of a vehicle, remaining time to destination, remaining time until next vehicle arrives, etc. 
    However, although there is also an increasing interest on gathering actual information regarding what users demand from the public transport and how they use their services, to the best of our knowledge, there are no final end-user tools that have tackled both the problems of: {\em (i)} effectively handling the huge amount of data that arises when we track all the user's trips (e.g. within a large city) and {\em (ii)} efficiently exploiting the underlying information.
    %Obtaining such  data necessarily required using some technology to track user's movements along the transport network. Currently, it is not uncommon to find cities having a smart card system \cite{smartCards00}, where each user has a personal public transport card with which he or she can pay for trips over several means of transport (bus, subway, etc) and also to switch between them. The use of these smart cards lets the transportation system to know where and when an individual started a trip, which provides valuable information for the administrators of the system. Indeed, the ending stop of a trip can be also estimated from boarding records alone \cite{wang2011review, alsger2016validating}.
    %Several works have also been presented regarding how travelers switch lines or where they finish their trips in order to completely track down any traveler \cite{daniil18,MUNIZAGA2012x}.  This new available information has been exploited in several scopes. For example, to gather the preferences of users on the chosen mean of public transportation to travel through a city \cite{publicTrans17}, to monitor urban traffic \cite{trafficMonit17}, or to study the traffic effects of the congestion charges introduced in some urban areas around the world \cite{congestion15}. However, to the best of our knowledge, there are no final end-user tools that have tackled both the problems of: {\em (i)} effectively handling the huge amount of data that arises when we track all the user's trips (e.g. within a large city) and {\em (ii)} efficiently exploiting the underlying information.
    
    %\ojo{Meter de alguna forma que tantos viajes, puntos de subida etc etc es big data y no hay ninguna olución final implementada ESCALABLE que permita tratar estos datos EN PROFUNDIDAD (HASTA AHORA que nosotros hemos resuelto el  puzzle)}
    
    
    Therefore, we are focusing on a solution to query and visualize historical records of aggregated patterns of movement from the travelers of a transportation network. While there exist popular solutions such as Google Maps to accurately calculate optimal multi-modal routes and even monitor the real time condition of the transportation networks, as indicated above, these systems only manage information about the {\em offer} of the network (such as schedules) and sometimes the real-time {\em demand} (current traffic conditions). In contrast, our approach aims at satisfying query needs about the historical demand over a large collection of past trajectories collected from users.
    
    %As we start handling and storing all those huge amounts of transport information we run into a popular problem nowadays: managing and exploiting big data. This problem has been analyzed  from several perspectives and a significant number of papers about this topic have been presented \cite{bigdataIOT,bigdataBIO}. In the context of public transport the issue is even worse due to the spatio-temporal nature of the knowledge. Spatio-temporal big data is a quite recent research topic that not only needs to deal with how a huge amount of spatial information should be stored and administered \cite{richter12} but also needs to review how this data should be displayed \cite{cortinas18,pelekis11}. Therefore, in this area we have to deal with two important problems: not only processing massive amounts of geographical information efficiently is required, but also developing user-friendly interfaces to access and visualize them are necessary \cite{7363974,LI2016119}.
    
    
    When dealing with vast amounts of historic traveler data, a valid first approach to tackle its analysis is to extend a traditional \gls{dbms} that could both assimilate and handle greater volumes of spatial data. A popular tool that fits in this description would be \textit{postGIS},\footnote{\url{https://postgis.net/}} a spatial extension for the widely adopted relational DBMS \textit{PostgreSQL}. 
    However, these solutions that include spatial indices are not the most adequate for our information needs, that are many times unrelated to a geographic component, but to the network elements and their relationships (``how many travelers started their trip at a stop X using line L to end it at a stop Y within the time frame T?''), while the large sizes of our data collections would force us to use solutions based on secondary memory, considerably affecting the performance of the system.
    
    %\ojo{Deberíamos falar de solucions distribuidas, que usen BD NoSQL,... botar un ollo a isto: \cite{XYGL2018}}
    
    
    
    %Apart from providing an efficient representation for the underlying data, there also exists the need for supporting visualization, exploitation, and management of spatial data by external users. The research in these topics is relatively new, meaning that, to our best knowledge, there does not exist a standard model of representation. Even though normally the users of applications that handle transportation-oriented data have advanced knowledge of geoinformatics, our target is to design a user-friendly interface that would not require of such skills, but only the minimum necessary understanding of the transportation domain.
    
    
    
    %In this work, we focus in the problem of managing data from a public transport system and we present a tool that includes a user-friendly interface to simplify the visualization and exploitation of the underlying data.  Our tool takes advantage of the recent \gls{xctr}\ structure presented in \cite{daniil18} that uses compact data structures to represent the data related to user's trips and efficiently solves some queries on aggregated data. This permitted us not only to save storage space but also boosted query performance when solving some important queries from a public transportation management system. 
    %To the best of our knowledge, it is the first tool that uses a representation of the data based on compact data structures.
    %While there exist other previous works that make use of compact data structures to efficiently index large collections of in-network trajectories \cite{brisaboa2018compact, koide2018cinct,  koide15}, the recent \gls{xctr}\ solution in \cite{daniil18} is novel as it implements a conceptual model that could adapt to any particular transportation system. To decrease the redundancy in the representation of user's trajectories, instead of listing all the stops traversed along a user's trip, \gls{xctr}\ stores only the stops where a user boards a vehicle along a whole trip that chains several lines, and the intermediate stops may be easily deduced from the network representation. 
    
    Given that we are focusing on a tool specifically for public transportation networks, our interface is built over the \gls{xctr} (introduced in the Section~\ref{sec:newctr:str:xctr}), which has allowed us to achieve fast response times in our tested use cases. In any case, our proposed tool is flexible enough to allow data to be represented in a more traditional way, such as with a relational \gls{dbms}.
	
	\section{Architecture}
	From a functional perspective and then a technological one.
	
	\section{API}
	A nice and easy way to make queries and return data.
	
	\section{User Interface}
	With a lot of screencaps that are going to take like 10 pages! Maybe even an evaluation section (which would require me to make it actually work well!).