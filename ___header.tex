% Packages --------------------------------------------------------------------

\usepackage{booktabs,multirow,multicol,array,colortbl,makecell,tabularx} % tablas


\usepackage[width=127mm,height=191mm,includehead,hcentering,vcentering]{geometry}
\usepackage{graphicx,subfigure}				% Include graphics
\usepackage{caption}	% Customize captions
\usepackage[T1]{fontenc} 			% Extended character set
\usepackage{lmodern, microtype}		% Load vector fonts; better text formatting
\usepackage{amsmath,amssymb,amsfonts,amsthm}	% Math support, example environment
\usepackage{fancyhdr} 				% Headers
\usepackage{xspace}	  				% Create macros with appropriate extra space after
\usepackage{lipsum}					% Fill sections
\usepackage{fixltx2e}				% Make math macros robust: \( \) \[ \]
\usepackage{afterpage}              % Manual control over footnotes



\usepackage[usenames,dvipsnames]{xcolor}	% Colors

\usepackage[chapter]{algorithm}
\usepackage{algorithmicx, algpseudocode}
\usepackage{float}

\usepackage{textcomp, gensymb}	% Common symbols (mu, degree)

%Debug packages
% \usepackage{refcheck}	% Show labels in output pdf. 
						% Note: local refcheck.sty has been modified to solve a bug in handling bibitems 
						% Note 2: since cleveref refcheck is unused




\usepackage{tikz}

\usepackage[breaklinks,colorlinks]{hyperref} % Hyperlinks in refs. Must be loaded at the end (except cleverref)
\usepackage{hypdvips}
% \usepackage[nameinlink]{cleveref}
\usepackage{cleveref}
\usepackage[acronym]{glossaries}    % Acronyms

% \usepackage{theorem}					% Theorems?
% \theoremstyle{plain}
% \newtheorem{definition}{Definition}

% \usepackage{lscape} 					% Set elements to landscape (e.g large figures, tables)
% \usepackage{setspace}					% ??
% \usepackage{rotating}
% \usepackage[tight]{subfigure}
% \usepackage{dcolumn}
% \usepackage[noline,linesnumbered,algochapter]{algorithm2e}
% \providecommand{\DontPrintSemicolon}{\dontprintsemicolon}
% \usepackage{algcompatible}
% \usepackage{sectsty}
%\usepackage[none]{hyphenat}
% \usepackage[tworuled,noline,linesnumbered,algochapter]{algorithm2e}


% FIXME: review in further versions
\hyphenpenalty=5000
\tolerance=1000
\widowpenalty=300
\clubpenalty=300
\emergencystretch=3em

\hfuzz=3em

\usepackage{etoolbox}
\apptocmd{\sloppy}{\hbadness 10000\relax}{}{}	% Avoid ''underfull hbox'' warnings in bibliography


% Separacion entre un parrafo y el siguiente
% \addtolength{\smallskipamount}{-7.60pt}
% \addtolength{\parskip}{.5\baselineskip}%{.5\baselineskip}

% \setlength{\smallskipamount}{-7.60pt}
% \setlength{\parskip}{\smallskipamount}

% Fuente de las subsubsecciones
%\subsubsectionfont{\bf}

\setcounter{tocdepth}{4} \setcounter{secnumdepth}{4}

% Fuentes de los pies de figura
\renewcommand{\captionlabelfont}{\bfseries}
\setlength{\captionmargin}{\parindent}

\renewcommand{\arraystretch}{1.2}	% Adjust vertical margins in tables

%Comandos

\newcounter{example}[chapter]
\def\theexample{\thechapter.\arabic{example}}
\newenvironment{example}{\par\noindent\textbf{Example \refstepcounter{example}\theexample:}}{}

% I need subsubsubsections, but not in the index
\newcommand{\subsubsubsection}[1]{\vspace{2mm}\par\noindent\textbf{#1}}

\renewcommand{\algorithmicrequire}{\textbf{Input:}}
\renewcommand{\algorithmicensure}{\textbf{Output:}}
\newcommand{\Input}{\Require}
\newcommand{\Output}{\Ensure}
\newcommand{\false}{\textbf{false}}
\newcommand{\red}[1]{{\color{red} #1 }}
\newcommand{\blue}[1]{{\color{blue} #1 }}
\newcommand{\cyan}[1]{{\color{cyan} #1 }}
\newcommand{\green}[1]{{\color{green} #1 }}
\newcommand{\todo}[1]{\blue{#1}}
\newcommand{\rewrite}[1]{\cyan{#1}}
\newcommand{\fixme}[1]{{\Large \red{#1}}}

\newcommand*{\irank}[0]{\emph{rank}\xspace}
\newcommand*{\iselect}[0]{\emph{select}\xspace}
\newcommand*{\iaccess}[0]{\emph{access}\xspace}

\DeclareMathOperator\rank{rank}
\DeclareMathOperator\select{select}
\DeclareMathOperator\access{access}

\renewcommand*{\k}[0]{\ensuremath{K}\xspace}
\newcommand*{\kk}[0]{\ensuremath{\k^2}\xspace}
\newcommand*{\kkk}[0]{\ensuremath{\k^3}\xspace}
\newcommand*{\KK}[0]{\ensuremath{\K^2}\xspace}
\newcommand*{\Ktree}[0]{\ensuremath{\k^2}-tree\xspace}
\newcommand*{\ktree}[0]{\ensuremath{\k^2}-tree\xspace}
\newcommand*{\ktrees}[0]{\ensuremath{\k^2}-trees\xspace}
\newcommand*{\Ktrees}[0]{\ensuremath{\k^2}-trees\xspace}
\newcommand*{\koct}[0]{\ensuremath{\k^3}-tree\xspace}
\newcommand*{\Koct}[0]{\ensuremath{\k^3}-tree\xspace}
\newcommand*{\kntree}[0]{\ensuremath{\k^n}-tree\xspace}
\newcommand*{\Kntree}[0]{\ensuremath{\k^n}-tree\xspace}
\newcommand{\kones}[0]{\ensuremath{\k^2}-tree1\xspace}
\newcommand{\koness}[0]{\ensuremath{\k^2}-tree1s\xspace}
\newcommand{\Kones}[0]{\kones}
\newcommand{\kdouble}[0]{\ensuremath{\k^2}-tree1\ensuremath{^{\mathrm{2bits-naive}}}\xspace}
\newcommand{\ksus}[0]{\ensuremath{\k^2}-tree1\ensuremath{^{\mathrm{2bits}}}\xspace}
\newcommand{\kdf}[0]{\ensuremath{\k^2}-tree1\ensuremath{^{\mathrm{df}}}\xspace}
\newcommand{\klevel}[0]{\ensuremath{\k^2}-tree1\ensuremath{^{\mathrm{1-5 bits}}}\xspace}
\newcommand{\Kdouble}[0]{\kdouble}
\newcommand{\Ksus}[0]{\ksus}
\newcommand{\Kdf}[0]{\kdf}
\newcommand{\Klevel}[0]{\klevel}
\newcommand{\btree}[0]{B-tree\xspace}
\newcommand{\btrees}[0]{B-trees\xspace}
\newcommand{\bplus}[0]{B\ensuremath{^+}-tree\xspace}
\newcommand{\bpluss}[0]{B\ensuremath{^+}-trees\xspace}
\newcommand{\dktree}[0]{d\ktree}
\newcommand{\dktrees}[0]{d\ktrees}
\newcommand{\Dktree}[0]{D\ktree}
\newcommand{\Dktrees}[0]{D\dtrees}
\newcommand{\iktree}{I\ktree}
\newcommand{\diktree}{diff-I\ktree}
\newcommand{\mktree}{M\ktree}

\newcommand{\kkind}{M\kones}
\newcommand{\kkacum}{AM\kones}
\newcommand{\ikones}{I\kones}
\newcommand{\koctindex}{\koct}

\newcommand{\diffk}[0]{diffK}
\newcommand{\ediffk}[0]{enh-diffK}
\newcommand{\Diffk}[0]{DiffK}
\newcommand{\Ediffk}[0]{Enh-diffK}

\newcommand{\ktreap}[0]{\ensuremath{\k^2}-treap\xspace}
\newcommand{\ktreaps}[0]{\ensuremath{\k^2}-treaps\xspace}

\newcommand{\ZERO}[0]{0\xspace}
\newcommand{\ZEROS}[0]{0's\xspace}
\newcommand{\ONE}[0]{1\xspace}
\newcommand{\ONES}[0]{1s\xspace}

\newcommand{\dash}[0]{\text{--}}

\newcommand{\kt}[0]{\ensuremath{k}\xspace}

\newcommand{\tdktree}[0]{tt\ktree}


\crefname{algorithm}{Algorithm}{Algorithms}
\crefname{example}{Example}{Examples}
\crefname{part}{Part}{Parts}
\crefname{table}{Table}{Tables}

% \normalfont
% \DeclareFontShape{T1}{lmr}{bx}{sc} { <-> ssub * cmr/bx/sc }{}

\normalfont
\DeclareFontShape{T1}{lmr}{bx}{sc} { <-> ssub * cmr/bx/sc }{}

%%%%%%%%%%%%%%%%%%%%%%%%%%%%%%%%%%%%%%%%%%%%%%%%%%%%%%%%%%%
%%%%%%%%%%%%%%%% Sandbox
% \DeclareSymbolFont{sfoperators}{OT1}{cmss}{m}{n}
% \DeclareSymbolFontAlphabet{\mathsf}{sfoperators}
% 
% \makeatletter
% % \def\operator@font{\mathgroup\symsfoperators}
% \def\select@font{\mathgroup\symsfoperators}
% \makeatother
% 
% \makeatletter
% \def\rank@font{\sf}
% \def\select@font{\bf}
% \makeatother

% \newcommand{\rank}[0]{\ensuremath{\mathrm{rank}}\xspace}
% \newcommand{\select}[0]{\ensuremath{\mathrm{select}}\xspace}
% \newcommand{\irank}[0]{\emph{rank}\xspace}
% \newcommand{\iselect}[0]{\emph{select}\xspace}
% \def\rank{\ensuremath{\mathsf{rank}}}
% \def\select{\ensuremath{\mathsf{select}}}
% \DeclareMathOperator\rank{rank}
% \DeclareMathOperator\select{select}

%%%%%%%%%%%%%% End sandbox
%%%%%%%%%%%%%%%%%%%%%%%%%%%%%%%%%%%%%%%%%%%%%%%%%%%%%
