% Packages --------------------------------------------------------------------

\usepackage{booktabs,multirow,multicol,array,colortbl,makecell,tabularx} % tablas
\newcommand{\STAB}[1]{\begin{tabular}{@{}c@{}}#1\end{tabular}}

\usepackage[width=127mm,height=191mm,includehead,hcentering,vcentering]{geometry}
\usepackage{graphicx,subfigure}				% Include graphics
\usepackage{caption}	% Customize captions
\usepackage[T1]{fontenc} 			% Extended character set
\usepackage{lmodern, microtype}		% Load vector fonts; better text formatting
\usepackage{amsmath,amssymb,amsfonts,amsthm}	% Math support, example environment
\usepackage{fancyhdr} 				% Headers
\usepackage{xspace}	  				% Create macros with appropriate extra space after
\usepackage{lipsum}					% Fill sections
%\usepackage{fixltx2e}				% Make math macros robust: \( \) \[ \]
\usepackage{afterpage}              % Manual control over footnotes



\usepackage[usenames,dvipsnames]{xcolor}	% Colors

\usepackage[chapter]{algorithm}
\usepackage{algorithmicx, algpseudocode}
\usepackage{float}

\usepackage{textcomp, gensymb}	% Common symbols (mu, degree)

%Debug packages
% \usepackage{refcheck}	% Show labels in output pdf. 
						% Note: local refcheck.sty has been modified to solve a bug in handling bibitems 
						% Note 2: since cleveref refcheck is unused




\usepackage{tikz}

\usepackage[breaklinks,colorlinks,driverfallback=dvipdfm]{hyperref} % Hyperlinks in refs. Must be loaded at the end (except cleverref)
\usepackage{hypdvips} % comment to remove error in pdflatex
% \usepackage[nameinlink]{cleveref}
\usepackage{cleveref}
\usepackage[nomain,acronym]{glossaries}    % Acronyms

% \usepackage{theorem}					% Theorems?
% \theoremstyle{plain}
% \newtheorem{definition}{Definition}

% \usepackage{lscape} 					% Set elements to landscape (e.g large figures, tables)
% \usepackage{setspace}					% ??
% \usepackage{rotating}
% \usepackage[tight]{subfigure}
% \usepackage{dcolumn}
% \usepackage[noline,linesnumbered,algochapter]{algorithm2e}
% \providecommand{\DontPrintSemicolon}{\dontprintsemicolon}
% \usepackage{algcompatible}
% \usepackage{sectsty}
%\usepackage[none]{hyphenat}
% \usepackage[tworuled,noline,linesnumbered,algochapter]{algorithm2e}


% FIXME: review in further versions
\hyphenpenalty=5000
\tolerance=1000
\widowpenalty=300
\clubpenalty=300
\emergencystretch=3em

\hfuzz=3em

\usepackage{etoolbox}
\apptocmd{\sloppy}{\hbadness 10000\relax}{}{}	% Avoid ''underfull hbox'' warnings in bibliography

% Separacion entre un parrafo y el siguiente
% \addtolength{\smallskipamount}{-7.60pt}
% \addtolength{\parskip}{.5\baselineskip}%{.5\baselineskip}

% \setlength{\smallskipamount}{-7.60pt}
% \setlength{\parskip}{\smallskipamount}

% Fuente de las subsubsecciones
%\subsubsectionfont{\bf}

\setcounter{tocdepth}{4} \setcounter{secnumdepth}{4}

% Fuentes de los pies de figura
\renewcommand{\captionlabelfont}{\bfseries}
\setlength{\captionmargin}{\parindent}

\renewcommand{\arraystretch}{1.2}	% Adjust vertical margins in tables

%Comandos

\newcounter{example}[chapter]
\def\theexample{\thechapter.\arabic{example}}
\newenvironment{example}{\par\noindent\textbf{Example \refstepcounter{example}\theexample:}}{}

% I need subsubsubsections, but not in the index
\newcommand{\subsubsubsection}[1]{\vspace{2mm}\par\noindent\textbf{#1}}

\renewcommand{\algorithmicrequire}{\textbf{Input:}}
\renewcommand{\algorithmicensure}{\textbf{Output:}}
\newcommand{\Input}{\Require}
\newcommand{\Output}{\Ensure}
\newcommand{\false}{\textbf{false}}
\newcommand{\red}[1]{{\color{red} #1 }}
\newcommand{\blue}[1]{{\color{blue} #1 }}
\newcommand{\cyan}[1]{{\color{cyan} #1 }}
\newcommand{\green}[1]{{\color{green} #1 }}
\newcommand{\todo}[1]{\blue{#1}}
\newcommand{\rewrite}[1]{\cyan{#1}}
\newcommand{\fixme}[1]{{\Large \red{#1}}}

\newcommand*{\irank}[0]{\emph{rank}\xspace}
\newcommand*{\iselect}[0]{\emph{select}\xspace}
\newcommand*{\iaccess}[0]{\emph{access}\xspace}
\newcommand*{\irange}[0]{\emph{range}\xspace}

\DeclareMathOperator\rank{rank}
\DeclareMathOperator\select{select}
\DeclareMathOperator\access{access}
\DeclareMathOperator\range{range}
\DeclareMathOperator\cnt{count}
\DeclareMathOperator\bsearch{bsearch}

\newcommand{\boardX}{\texttt{board\_X}}
\newcommand{\alightX}{\texttt{alight\_X}}
\newcommand{\useL}{\texttt{use\_L}}
\newcommand{\boardT}{\texttt{board\_T}}
\newcommand{\alightT}{\texttt{board\_T}}
\newcommand{\loadX}{\texttt{load\_X}}
\newcommand{\startX}{\texttt{start\_X}}
\renewcommand{\endX}{\texttt{end\_X}}
\newcommand{\switchX}{\texttt{switch\_X}}
\newcommand{\XtoY}{\texttt{from\_X\_to\_Y}}
\newcommand{\startL}{\texttt{start\_L}}
\renewcommand{\endL}{\texttt{end\_L}}
\newcommand{\startT}{\texttt{start\_T}}
\renewcommand{\endT}{\texttt{end\_T}}
\newcommand{\loadT}{\texttt{load\_T}}
\newcommand{\tripT}{\texttt{trip\_T}}
\newcommand{\topK}{\texttt{top\_K}}

\newenvironment{code}
{ % \scriptsize
    %\fontsize{8}{10}\selectfont
    \begin{tabbing}
    \rule{\textwidth}{0.25mm} \\
    xxxx\=xxxx\=xxxx\=xxxx\=xxxx\=xxxx\=xxxx\=xxxx \kill}
    {\rule{\textwidth}{0.25mm}
    \end{tabbing}
}

\crefname{algorithm}{Algorithm}{Algorithms}
\crefname{example}{Example}{Examples}
\crefname{part}{Part}{Parts}
\crefname{table}{Table}{Tables}

% \normalfont
% \DeclareFontShape{T1}{lmr}{bx}{sc} { <-> ssub * cmr/bx/sc }{}

\normalfont
\DeclareFontShape{T1}{lmr}{bx}{sc} { <-> ssub * cmr/bx/sc }{}
