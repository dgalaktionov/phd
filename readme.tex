% Original author: Christoph Loitzenbauer (VDA Group University of Vienna)
% Change log: 
% 04.02.2019 - first draft
%
\documentclass{article}
\usepackage[utf8]{inputenc}
\usepackage{hyperref}
\usepackage{geometry}

\title{Doctoral Thesis Template Readme}
\date{February 2019}




\begin{document}

\maketitle

\section{Using the template}
You should not change the layout of the title page. The layout of the rest of the pages can be changed to fit your thesis.
The titlepage is based on the article class. So if you want to use a different class without changing the title page I would recommend using the template for creating the title page and then use it in your project using the 
\verb|\includepdf{<filename>}| from the \verb|pdfpages| package. \newline 


Within the template there is a class provided that generates the title page and provides commands to change it. 
There is also an example tex and bib file that uses the class. 

\section{Embed all fonts in pdf}
Please make sure that you embed all fonts in your pdf. Also make sure all the fonts of any figures that were used in the document are embedded. 
If you don't use any pdf figures pdfLaTeX should embed all fonts automatically.

\subsection{Linux}
On Linux you can use the command \begin{verbatim}
    pdffonts my_file.pdf
\end{verbatim}
to check if the fonts are embbeded. Check if all the fonts listed have a "yes" in the "emb" column. 
\begin{verbatim}
name                                 type              encoding         emb sub uni 
------------------------------------ ----------------- ---------------- --- --- --- 
BXJBCJ+NimbusSanL-Bold               Type 1            Custom           yes yes no     
HEMYJL+NimbusSanL-Regu               Type 1            Custom           yes yes no     
OOJWDR+SFRM1000                      Type 1            Custom           yes yes no      
OHLNOC+SFRM0900                      Type 1            Custom           yes yes no   
...
\end{verbatim}
For more information see 

\url{https://www.karlrupp.net/2016/01/embed-all-fonts-in-pdfs-latex-pdflatex/}
\subsection{Windows}
On Windows using the Adobe Acrobat Reader the fonts can be found at
\begin{verbatim}
File > Properties > Fonts
\end{verbatim}
For more information please consult
\begin{enumerate}
\item \url{https://helpx.adobe.com/acrobat/using/pdf-fonts.html}

\item \url{https://www.overleaf.com/learn/latex/Questions/My_submission_was_rejected_by_the_journal_because_%22Font_XYZ_is_not_embedded%22._What_can_I_do%3F} 
\end{enumerate}

\section{Making a PDF/A-1 compatible pdf}
% Making the the work PDF/A compatible:
%"Für die Ablieferung Ihrer Abschlussarbeit in elektronischer Form sind als einziges Dateiformat PDF-Dokumente in der von Adobe spezifizierten Version PDF/A-1 bzw. PDF/A-2 erlaubt."
% https://hopla.univie.ac.at/erstellen_von_pdf.pdf
As can be seen in \url{https://hopla.univie.ac.at/erstellen_von_pdf.pdf} and the "Informationen zur Erstellung und Abgabe von Hochschulschriften" it is required to provide a PDF/A-1 or PDF/A-2 version of your thesis.

\subsection{validation}
To validate the produced PDF you can either use the Preflight tool included in Adobe Acrobat Pro or a free online version. E.g. \url{https://www.pdf-online.com/osa/validate.aspx}.
Please take caution as different validation tools can report different results.

\subsection{pdfx}
First step to get PDF/A combatibility is by using the package \verb|pdfx| - make sure it is included before the hyperref package.
\begin{verbatim}
\usepackage[a-1b]{pdfx}
\end{verbatim}
This package is already included in the class.
For more details about the package and the following steps please consult \url{http://texdoc.net/texmf-dist/doc/latex/pdfx/pdfx.pdf}

\subsection{metadata}
There is a section in the beginning of the .tex file where you can change the metadata.
Change your Name, Title, Subject,Keywords and remove/add Information as you like. To find out what fields are possible please check here. \url{http://texdoc.net/texmf-dist/doc/latex/pdfx/pdfx.pdf#subsection.2.3}

A file containing the metadata is created when compiling. 

\subsection{figures}
As already mentioned, make sure that all the fonts used in the pictures are included. Furthermore transparency in pictures causes issues, please convert transparent figures into their nontransparent version. 

Using Linux the command \verb|pdfimages -list <pdf>| shows the typpe of all images used. The type should always be \verb|image| and not \verb|smask|. Check and convert these images.

Additionally there can be problems if figures use different color spaces. Use the same command as before and check if all images use the same color. 

If color is really important in your work it might also be a good idea to use an ICC profile for the color. 
For more details about colors check \url{http://texdoc.net/texmf-dist/doc/latex/pdfx/pdfx.pdf#subsection.2.5}

It is also possible to convert the pictures automatically using ghostscript. But always check the results manually. 

\subsection{Other errors}
Due to the complexity of Latex files there can be many more errors that are not covered in the readme.

The Preflight tool included in Adobe Acrobat Pro also has the ability to fix some errors. For example EOL (End of Line) errors can be fixed with its analyize and fix option. 
Please also check if any of the following pages might have a solution to your problem:
\begin{enumerate}
    \item \url{https://www.mathstat.dal.ca/~selinger/pdfa/}
    \item \url{https://blog.zhaw.ch/icclab/creating-pdfa-documents-for-long-term-archiving/}
    \item german: \url{http://kulturreste.blogspot.com/2014/06/grrrr-oder-wie-man-mit-latex-vielleicht.html}
    \item \url{https://support.stmdocs.in/wiki/?title=Generating_PDF/A_compliant_PDFs_from_pdftex}
    \item \url{http://texdoc.net/texmf-dist/doc/latex/pdfx/pdfx.pdf}
\end{enumerate}

\subsection{Tagged PDF}
Currently with Latex it is only possible to create files that are in the PDF/A-1b format. The PDF/A-1a format required the PDF to be tagged which is currently not possible in a satisfactory way.
A manual tagging with Adobe Acrobat Pro is possible but not recommended.

More information about the current status of tagged pdfs can be found here:

\url{https://umij.wordpress.com/2016/08/11/the-sad-state-of-pdf-accessibility-of-latex-documents/} 

and here

\url{https://www.tug.org/TUGboat/tb30-2/tb95moore.pdf}

\section{General Remarks}
Highest priority should always be the embedding of all fonts. Further compliance with the PDF/A standards is always desired, but talk to your supervisor in any case.

Highest priority should always be the embedding of all fonts. Further compliance with the PDF/A standards is always desired, but talk to your supervisor in any case.

Please also check the following resources if you have problems and need assistance

\begin{enumerate}
\item \url{https://spl29.univie.ac.at/fileadmin/user_upload/s_spl29/Studium/abschluss_master/Infoblatt_Hochschulschriften.pdf} 
\item \url{https://e-theses.univie.ac.at/E-Theses_erstellen_von_pdf.pdf}
\end{enumerate}


\end{document}